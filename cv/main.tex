%-------------------------
% Resume in Latex
% Author : Jake Gutierrez
% Based off of: https://github.com/sb2nov/resume
% License : MIT
%------------------------

\documentclass[letterpaper,11pt]{article}

\usepackage{latexsym}
\usepackage[empty]{fullpage}
\usepackage{titlesec}
\usepackage{marvosym}
\usepackage[usenames,dvipsnames]{color}
\usepackage{verbatim}
\usepackage{enumitem}
\usepackage[hidelinks]{hyperref}
\usepackage{fancyhdr}
\usepackage[english]{babel}
\usepackage{tabularx}
\usepackage{xcolor}
\input{glyphtounicode}
\usepackage{enumerate}


%----------FONT OPTIONS----------
% sans-serif
% \usepackage[sfdefault]{FiraSans}
% \usepackage[sfdefault]{roboto}
% \usepackage[sfdefault]{noto-sans}
% \usepackage[default]{sourcesanspro}

% serif
% \usepackage{CormorantGaramond}
% \usepackage{charter}


\pagestyle{fancy}
\fancyhf{} % clear all header and footer fields
\fancyfoot{}
\renewcommand{\headrulewidth}{0pt}
\renewcommand{\footrulewidth}{0pt}

% Adjust margins
\addtolength{\oddsidemargin}{-0.5in}
\addtolength{\evensidemargin}{-0.5in}
\addtolength{\textwidth}{1in}
\addtolength{\topmargin}{-.5in}
\addtolength{\textheight}{1.0in}

\urlstyle{same}

\raggedbottom
\raggedright
\setlength{\tabcolsep}{0in}

% Sections formatting
\titleformat{\section}{
  \vspace{-4pt}\scshape\raggedright\large
}{}{0em}{}[\color{black}\titlerule \vspace{-5pt}]

% Ensure that generate pdf is machine readable/ATS parsable
\pdfgentounicode=1

%-------------------------
% Custom commands
\newcommand{\resumeItem}[1]{
  \item\small{
    {#1 \vspace{-2pt}}
  }
}

\newcommand{\resumeSubheading}[4]{
  \vspace{-2pt}\item
    \begin{tabular*}{0.97\textwidth}[t]{l@{\extracolsep{\fill}}r}
      \textbf{#1} & #2 \\
      \textit{\small#3} & \textit{\small #4} \\
    \end{tabular*}\vspace{-7pt}
}

\newcommand{\resumeSubSubheading}[2]{
    \item
    \begin{tabular*}{0.97\textwidth}{l@{\extracolsep{\fill}}r}
      \textit{\small#1} & \textit{\small #2} \\
    \end{tabular*}\vspace{-7pt}
}

\newcommand{\resumeProjectHeading}[2]{
    \item
    \begin{tabular*}{0.97\textwidth}{l@{\extracolsep{\fill}}r}
      \small#1 & #2 \\
    \end{tabular*}\vspace{-7pt}
}

\newcommand{\resumeSubItem}[1]{\resumeItem{#1}\vspace{-4pt}}

\renewcommand\labelitemii{$\vcenter{\hbox{\tiny$\bullet$}}$}

\newcommand{\resumeSubHeadingListStart}{\begin{itemize}[leftmargin=0.15in, label={}]}
\newcommand{\resumeSubHeadingListEnd}{\end{itemize}}
\newcommand{\resumeItemListStart}{\begin{itemize}}
\newcommand{\resumeItemListEnd}{\end{itemize}\vspace{-5pt}}

\definecolor{darkgreen}{HTML}{228833}

%-------------------------------------------
%%%%%%  RESUME STARTS HERE  %%%%%%%%%%%%%%%%%%%%%%%%%%%%


\begin{document}


\begin{center}
    \textbf{\Huge \scshape Krijn Doekemeijer} \\ \vspace{1pt}
    \small Vrije Universiteit Amsterdam, De Boelelaan 1111, 1081 HV Amsterdam, The Netherlands \\ \vspace{1pt}
    \href{https://krien.github.io/}{\underline{krien.github.io}} $|$
    \href{https://linkedin.com/in/krijn-doekemeijer-9692801aa}{\underline{linkedin.com/in/krijn-doekemeijer}} $|$
    \href{https://github.com/Krien}{\underline{github.com/Krien}} \\
    \textit{ Computer Science PhD student, improving performance/energy QoS of storage- and networked systems. } 
\end{center}

%-----------EXPERIENCE-----------
\section{Employment}
  \resumeSubHeadingListStart
    \resumeSubheading
      {Doctor of Philosophy in Computer Science}{Dec. 2022 --}
      {Vrije Universiteit Amsterdam (VU)}{Amsterdam, The Netherlands}
      \resumeItemListStart
      \resumeItem{ 
        Advisors: Animesh Trivedi, Balakrishnan Chandrasekaran
      }
      \resumeItem{ 
        Research team: Massivizing Computer Systems Group (AtLarge)
      }
      \resumeItem{
        Research focus: Computer systems, specializing in performance/energy QoS for storage- and networked systems
      }
      \resumeItemListEnd
    \resumeSubheading
      {Software developer for the Customer Experience (CX) team}{Oct. 2020 -- Nov. 2021}
      {Kaartje2Go, Working Talent}{Zwolle, The Netherlands}
      \resumeItemListStart
        \resumeItem{ 
          Focus: Setting up the Analytics pipeline, telemetry tooling and A/B test tooling
        }
        \resumeItem{
          Technologies: Full-stack web development, analytics (GA, GTM...), genetic algorithms, DevOPs (AWS, CI/CD)
        }
      \resumeItemListEnd

  \resumeSubHeadingListEnd

%-----------EDUCATION-----------
\section{Education}
\resumeSubHeadingListStart
    \resumeSubheading
    {Vrije Universiteit Amsterdam and Universiteit van Amsterdam}{Amsterdam, The Netherlands}
    {Joint Masters degree in Computer Science (Big Data Engineering track). GA 8.9}{Sep. 2020 -- Aug. 2022}
  \resumeSubheading
    {Utrecht University}{Utrecht, The Netherlands}
    {Bachelors degree in Computer Science (Gametech specialization). GA 8.6 }{Sep. 2017 -- May 2020}
\resumeSubHeadingListEnd





\section{Research}

\textbf{Refereed publications}
\begin{itemize}[label={}]

\item
  ``PowerSensor3: A fast and accurate open source power measurement tool'', 
  S. van der Vlugt, L. Oostrum, G. Schoonderbeek, B. van Werkhoven, B. Veenboer, \underline{K. Doekemeijer}, J. Romein,
  \textit{2025 IEEE International Symposium on Performance Analysis of Systems and Software (ISPASS)}, 
  TBD, 2025.

\item
  ``Columbo: A Reasoning Framework for Kubernetes’ Configuration Space'', 
  M. Jansen, S. Talluri, \underline{K. Doekemeijer}, N. Tehrany, A. Iosup, A. Trivedi,
  \textit{In Proceedings of 16th ACM/SPEC International Conference on Performance Engineering (ICPE'25)}, 
  TBD, 2025.

\item
  ``An I/O Characterizing Study of Offloading LLM Models and KV Caches to NVMe SSD'', 
  Z. Ren, \underline{K. Doekemeijer}, T. De Matteis,  C. Pinto, R. Stoica, A. Trivedi,
  \textit{In Proceedings of the 5th Workshop on Challenges and Opportunities of Efficient and Performant Storage Systems (CHEOPS'25)}, 
  TBD, 2025.

\item
  ``Exploring I/O Management Performance in ZNS with ConfZNS++'', 
  \underline{K. Doekemeijer}, D. Maisenbacher, Z. Ren, N. Tehrany, M. Bjørling, A. Trivedi,
  \textit{In Proceedings of the 17th ACM International Conference on Systems and Storage (SYSTOR'24)}, 
  September, 2024.

\item
  ``ZWAL: Rethinking Write-ahead Logs for ZNS SSDs with Zone Appends'', 
  \underline{K. Doekemeijer}, Z. Ren, N. Tehrany, A. Trivedi,
  \textit{ACM SIGOPS Operating Systems Review, Volume 58, Issue 1 (SIGOPS OSR'24)}, 
  August, 2024.

\item
  ``zns-tools: An eBPF-powered, Cross-Layer Storage Profiling Tool for NVMe ZNS SSDs'', 
  N. Tehrany, \underline{K. Doekemeijer}, A. Trivedi,
  \textit{In Proceedings of the 4th Workshop on Challenges and Opportunities of Efficient and Performant Storage Systems (CHEOPS'24)},
  May, 2024.

\item
  ``ZWAL: Rethinking Write-ahead Logs for ZNS SSDs with Zone Appends'', 
  \underline{K. Doekemeijer}, Z. Ren, N. Tehrany, A. Trivedi,
  \textit{In Proceedings of the 4th Workshop on Challenges and Opportunities of Efficient and Performant Storage Systems (CHEOPS'24)}, 
  May, 2024.

\item
  ``BFQ, Multiqueue-Deadline, or Kyber? Performance Characterization of Linux Storage Schedulers in the NVMe Era'', 
  Z. Ren, \underline{K. Doekemeijer}, N. Tehrany, A. Trivedi,
  \textit{In Proceedings of 15th ACM/SPEC International Conference on Performance Engineering (ICPE'24)}, 
  May, 2024. 
  \textbf{Winner of ``Best research paper award''}.

\item
  ``A Systematic Configuration Space Exploration of the Linux Kyber I/O Scheduler'', 
  Z. Ren, \underline{K. Doekemeijer}, A. Trivedi,
  \textit{Companion of the 15th ACM/SPEC International Conference on Performance Engineering (ICPE’24 Companion)}, 
  May, 2024.

\item
  ``Reviving Storage Systems Education in the 21st Century — An experience report'', 
  A. Trivedi, M. Jansen, \underline{K. Doekemeijer}, S. Talluri, N. Tehrany,
  \textit{In 2024 IEEE/ACM 24rd International Symposium on Cluster, Cloud and Internet Computing (CCGrid'24)}, 
  May, 2024. 
  \textbf{Nominated for ``Best paper award''}.

\item
  ``Performance Characterization of NVMe Flash Devices with Zoned Namespaces (ZNS)'', 
  \underline{K. Doekemeijer}, N. Tehrany, B. Chandrasekaran, M. Bjørling, A. Trivedi,
  \textit{2023 IEEE International Conference on Cluster Computing (CLUSTER)}, 
  October, 2023.

\end{itemize}

\textbf{Preprints}
\begin{itemize}[label={}]
\item
  ``Understanding (Un)Written Contracts of NVMe ZNS Devices with \texttt{zns-tools}'', 
  N. Tehrany, \underline{K. Doekemeijer}, A. Trivedi,
  \textit{arXiv:2307.11860; Computing Research Repository (CoRR)}, 
  July, 2023.

\item
  ``A Survey on the Integration of NAND Flash Storage in the Design of File Systems and the Host Storage Software Stack'', 
  N. Tehrany, \underline{K. Doekemeijer}, A. Trivedi,
  \textit{arXiv:2307.11866; Computing Research Repository (CoRR)}, 
  July, 2023.

\item
  ``Key-Value Stores on Flash Storage Devices: A Survey'', 
  \underline{K. Doekemeijer}, A. Trivedi,
  \textit{arXiv:2205.07975; Computing Research Repository (CoRR)}, 
  August, 2022.
\end{itemize}


\textbf{Other}
\begin{itemize}[label={}]
\item
  \textbf{Poster}: 
  ``Performance Characterization of NVMe Devices with Zoned Namespaces (ZNS)'',
  \underline{K. Doekemeijer}, N. Tehrany, B. Chandrasekaran, A. Trivedi,
  \textit{ICT.OPEN} (National Dutch ICT conference),
  April, 2024, Utrecht, The Netherlands.

\item
  \textbf{Poster}: 
  ``TropoDB: Design, Implementation and Evaluation of a KV-Store for Zoned Namespace Devices'',
  \underline{K. Doekemeijer}, A. Trivedi,
  \textit{ICT.OPEN} (National Dutch ICT conference),
  April, 2023, Utrecht, The Netherlands.

\item
  \textbf{Master thesis}: 
  ``TropoDB: Design, Implementation and Evaluation of an Optimised KV-Store for NVMe Zoned Namespace Devices'',
  August, 2022, Amsterdam, The Netherlands.
  Won the \textbf{ADS thesis award} (Amsterdam Data Science).
\end{itemize}

%-----------PROJECTS-----------
\section{Research Projects}
    \resumeSubHeadingListStart

    \resumeProjectHeading
      {\textbf{ConfZNS++} $|$ \emph{ZNS SSDs, emulator}} {April 2024 -- }
      \resumeItemListStart
        \resumeItem{ConfZNS++ is the first function-accurate emulator for ZNS that incorporates realistic I/O management operations.
        See: \href{https://github.com/stonet-research/confznsplusplus}{\underline{https://github.com/stonet-research/confznsplusplus}}.   }
      \resumeItemListEnd
    \resumeProjectHeading
      {\textbf{ZINC} $|$ \emph{ZNS SSDs, I/O scheduler}} {Oktober 2023 -- }
      \resumeItemListStart
        \resumeItem{ZINC is an I/O scheduler for ZNS with I/O management operations (i.e., reset and finish) as first-class citizens.
        See: \href{https://github.com/stonet-research/zinc-schedulers}{\underline{https://github.com/stonet-research/zinc-scheduler}}.   }
      \resumeItemListEnd
    \resumeProjectHeading
      {\textbf{ZWAL} $|$ \emph{ZNS SSDs, KV-store, WAL}} {July 2023 -- }
      \resumeItemListStart
        \resumeItem{ZWAL is a write-ahead log (WAL) redesigned to make use of the ZNS SSD append operation. It intents to leverage 
        the performance stability of ZNS, but keep throughput at high levels with the append operation. It is build on top of SZD.
        See \href{https://github.com/stonet-research/zwal}{\underline{https://github.com/stonet-research/zwal}}. }
      \resumeItemListEnd
    \resumeProjectHeading
      {\textbf{zns-tools} $|$ \emph{ZNS SSDs, tracing, profiling, end-to-end}} {March 2023 -- }
      \resumeItemListStart
        \resumeItem{zns-tools is the first step to make an end-to-end tool to trace storage requests from application to storage. For now it traces
        F2FS and Btrfs activity on ZNS, but we plan to extend it. See \href{https://github.com/stonet-research/zns-tools}{\underline{https://github.com/stonet-research/zns-tools}}. }
      \resumeItemListEnd
    \resumeProjectHeading
      {\textbf{TropoDB} $|$ \emph{ZNS SSDs, Key-value store, SPDK, raw storage}} {February 2022 --}
      \resumeItemListStart
        \resumeItem{ TropoDB is an ongoing reseach project of redesigning LSM-tree KV-stores for flash SSDs. It is an implementation in user-space (using SPDK) 
        that is originally build around ZNS SSDs. The intent is to provide a research project for each component of the LSM-tree.
        See: \href{https://github.com/Krien/TropoDB}{\underline{github.com/Krien/TropoDB}}. }
      \resumeItemListEnd
    \resumeProjectHeading
      {\textbf{Simple ZNS Device (SZD)} $|$ \emph{ZNS SSDs, SPDK, io\_uring, API}} {February 2022 -- }
      \resumeItemListStart
        \resumeItem{SZD is a storage engine for NVMe (ZNS) SSDs. It uses an opinionated subset of SPDK
        and io\_uring. See \href{https://github.com/Krien/SimpleZNSDevice}{\underline{github.com/Krien/SimpleZNSDevice}}.  }
      \resumeItemListEnd
        % \resumeProjectHeading
        %   {\textbf{File system for ZNS SSDs} $|$ \emph{C, C++, CMake, ZNS SSDs}} {November 2021 -- December 2021}
        %   \resumeItemListStart
        %     \resumeItem{For the university course ``Storage Systems", I worked on building a file system on top of a \textit{Flash Translation Layer} (FTL) made for ZNS devices. This file system was then tested and benchmarked with the key-value store RocksDB.}
        %   \resumeItemListEnd
        % \resumeProjectHeading
        %   {\textbf{Flash Translation Layer (FTL) for ZNS SSDs} $|$ \emph{C, C++, CMake, ZNS SSDs}} {November 2021 -- November 2021}
        %   \resumeItemListStart
        %     \resumeItem{For the university course ``Storage Systems", I worked on building a \textit{Flash Translation Layer} (FTL) directly on top of a ZNS device with the help of libnvme. Most of the project was written in C, with a bit of C++.}
        %   \resumeItemListEnd
    %     % \resumeProjectHeading
    %     %   {\textbf{Exidy Sorcerer literature study} $|$ \emph{Literature study, History, Emulators}} {January 2021 -- February 2021}
    %     %   \resumeItemListStart
    %     %     \resumeItem{For the university course ``History of Digital Cultures", I worked with a team on documenting the history of how an old Home Computer, known as the Exidy Sorcerer, was used in the Netherlands. I mainly worked on getting old Dutch software running in Microsoft BASIC to work in a Sorcerer emulator and documenting this software. }
    %     %   \resumeItemListEnd
    %     % \resumeProjectHeading
    %     %   {\textbf{Reddit visualisation} $|$ \emph{Python, JavaScript, Frontend, Machine Learning}} {February 2021 -- March 2021}
    %     %   \resumeItemListStart
    %     %     \resumeItem{For the university course ``Information Visualisation", I worked with a team on creating a tool to do sentiment analysis of Reddit. My tasks mainly working on various data processing and visualisation tasks; tools such as Bokeh, Pandas and jQuery were used to achieve this.}
    %     %   \resumeItemListEnd
    %     \resumeProjectHeading
    %       {\textbf{COVID-19 Twitter visualisation} $|$ \emph{Python, Machine learning, NLP}} {November 2020 -- December 2020}
    %       \resumeItemListStart
    %         \resumeItem{For the university course ``Web Data Processing", I worked with an enthusiastic team on a visualisation of the most important topics on Twitter during the COVID-19 pandemic for each country. For me, this mainly involved the topic modelling aspects (NLP, ML, Python).}
    %       \resumeItemListEnd
    %     \resumeProjectHeading
    %       {\textbf{COVID-19 Pollution map} $|$ \emph{Spark, Python, Big Data}} {September 2020 -- Oktober 2020}
    %       \resumeItemListStart
    %         \resumeItem{For the university course ``Large Scale Data Engineering", I worked with a team on a large scale data processing pipeline and visualisation tool of air pollution during the COVID-19 Pandemic. I focused on the data processing pipeline parts with Python, Apache Spark and DataBricks.}
    %       \resumeItemListEnd
    % % \resumeProjectHeading
    % %       {\textbf{Ray-tracing porting efforts} $|$ \emph{C\#, Java, C++, SDL}}{July 2020 -- July 2020}
    % %       \resumeItemListStart
    % %         \resumeItem{Porting a Raytracer I had originally written in C\# to Java and C++ to properly learn the languages.}
    % %       \resumeItemListEnd
    %   \resumeProjectHeading
    %       {\textbf{Haskell Shoot 'em up} $|$ \emph{Haskell, game development}}{Oktober 2018 -- November 2018}
    %       \resumeItemListStart
    %         \resumeItem{I developed a Shoot 'em up game with Haskell and Gloss (graphics library) for the course ``Functional programming". See \href{https://github.com/Krien/Haskell.ShootEmUp}{\underline{github.com/Krien/Haskell.ShootEmUp}}.}
    %       \resumeItemListEnd
    %    \resumeProjectHeading
    %       {\textbf{Noxium} $|$ \emph{Unity, C\#, game development}}{November 2017 -- Februari 2018}
    %       \resumeItemListStart
    %         \resumeItem{Developing a 3d beat 'em up game in the game engine Unity with a team of 4. For this project, I created the AI, UI and I/O logic,  menu and various multiplayer aspects of the game. }
    %       \resumeItemListEnd
    \resumeSubHeadingListEnd


\section{Talks}
    \textbf{Conference}
    \begin{itemize}[label={}]
    \item
        ``Exploring I/O Management Performance in ZNS with ConfZNS++'',
        \textit{SYSTOR'24}, September, 2024, Virtual, Israel.
    \item
        ``ZWAL: Rethinking Write-ahead Logs for ZNS SSDs with Zone Appends'',
        \textit{CHEOPS'24 at EuroSys 2024}, April, 2024, Athens, Greece.
    \item
        ``Performance Characterization of NVMe Flash Devices with Zoned Namespaces (ZNS)'',
        \textit{CLUSTER'23}, November, 2023, Santa Fe, NM, USA.
    \end{itemize}
    
    \textbf{Other}
    \begin{itemize}[label={}]
    \item
      ``ZWAL: Rethinking Write-ahead Logs for ZNS SSDs with Zone Appends'',
      \textit{CompSys'24} (Dutch Computer Systems), May, 2024, Sint michielsgestel, The Netherlands.
    
    \item
      ``TropoDB: Design, Implementation and Evaluation of a KV-Store for Zoned Namespace Devices'',
      \textit{CompSys'23} (Dutch Computer Systems), June, 2023, Soesterberg, The Netherlands.
    \end{itemize}

\section{Community Service}
\textbf{Reviewer}:
\begin{itemize}
\itemsep0em
\item Paper: ACM/SPEC International Conference on Performance Engineering (ICPE) -- 2024
\item Paper: IEEE/ACM International Symposium on Cluster, Cloud and Internet Computing (CCGrid) -- 2023--2024
\item Artifact: Symposium on Operating Systems Principles (SOSP) -- 2023
\end{itemize}

\section{Teaching, Supervision}
\textbf{Teaching Assistant}:
\begin{itemize}[label={}]
\itemsep0em
  \item Distributed Systems (MSc., 2024) -- Organizer for storage system lab projects\\
  \textit{Teaching students how to design, implement and evaluate distributed systems}.
  
  \item Advanced Network Programming (BSc., 2023--2024) -- Lead TA\\
  \textit{Teaching students how to design, implement and evaluate network stacks (i.e., ICMP, TCP)}.

  \item Storage Systems (MSc., 2023) -- Lead TA\\
  \textit{Teaching students how to design, implement and evaluate storage systems (i.e., FTL, FS, KV-store)}.

  \item Systems Seminar (MSc., 2023--2024) -- TA, lead TA for artifacts\\
  \textit{Teaching students how to read/review computer systems papers and how to reproduce/review paper artifacts}.
\end{itemize}

\textbf{Supervision}:
\begin{itemize}
\itemsep0em
  \item Gleb Mishchenko  (BSc. honours program):  TBD
  
  \item Sudarsan Sivakumar (MSc. thesis): 
  \textit{Performance Characterization Study of NVMe Storage Over TCP} (2024).
  
  \item Darko Vujica (BSc. thesis): 
  \textit{Exploring Redis Persistence Modes: Introducing AOFURing, an io\_uring AOF Extension} (2024).
  
  \item Vincent Kohm (BSc. thesis): 
  \textit{Optimizing Metadata Handling with vkFS: A Hybrid Key-Value Store File System leveraging RocksDB} (2024).
  
  \item Sudarsan Sivakumar (MSc. survey): 
  \textit{A survey on flash storage disaggregation: performance and quality of service considerations} (2024).
\end{itemize}

%
%-----------PROGRAMMING SKILLS-----------
% \section{Skills}
%  \begin{itemize}[leftmargin=0.15in, label={}]
%     \small{\item{
%      \textbf{Programming languages}{: Experience in C, C++, C\#, Python, JavaScript, TypeScript, PHP, Haskell and R. Knowledgeable about various other languages such as RUST, Matlab, a few Lisp dialects and APL. Proficient in the ``DSLs" Bash, HTML, SQL, LaTeX and CMake} \\
%      \textbf{Languages}{: Fluent in English and Dutch; can comprehend a little German} \\
%      \textbf{Frameworks/Libraries}{: Among others SPDK, BPF, RocksDB, LevelDB, LightGBM, TensorFlow, NLTK, MPI, Google Analytics suite, MonoGame, Gloss, SDL, OpenGL,  React, Cake, Symfony, Flask, Bokeh} \\
%      \textbf{Developer Tools}{: Git, Docker, build tools, QEMU, AWS, Kubernetes, Apache Spark, DataBricks, Linux, Windows, Unity, WSL,  BPFTrace, perf, fio} \\
%      \textbf{General skills}{: Machine learning, CI/CD, virtualisation, software testing, software architecture, game development, scrum/agile }
%      }} 
%  \end{itemize}


%-------------------------------------------
\end{document}